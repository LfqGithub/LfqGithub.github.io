\documentclass[tikz]{standalone}
\usetikzlibrary{datavisualization}
\usetikzlibrary{datavisualization.formats.functions}
\begin{document}
\begin{tikzpicture}

\pgfooclass{circle visualizer}
  {
	% Stores the name of the visualizer. This is needed for filtering and configuration
	\attribute name;
	% The constructor. Just setup the attribute.
	\method circle visualizer(#1) { \pgfooset{name}{#1} }
	% Connect to visualize signal.
	\method default connects() {
	  \pgfoothis.get handle(\me)
	\pgfkeysvalueof{/pgf/data visualization/obj}.connect(\me,visualize,visualize datapoint signal)
}
% This method is invoked for each data point. It checks whether the data point belongs to the correct
% visualizer and, if so, calls the macro \dovisualization to do the actual visualization.
\method visualize() {
\pgfdvfilterpassedtrue
\pgfdvnamedvisualizerfilter
\ifpgfdvfilterpassed
\dovisualization
\fi
}
}

\def\dovisualization{
  \pgfkeysvalueof{/data point/\pgfoovalueof{name}/execute at begin}
  \pgfpathcircle{\pgfpointdvdatapoint}{\pgfkeysvalueof{/data point/radius}}
  % \pgfusepath is done by |execute at end|
  \pgfkeysvalueof{/data point/\pgfoovalueof{name}/execute at end}
}

\tikzdatavisualizationset{
	visualize as circle/.style={
	  new object={
		when=after survey,
		store=/tikz/data visualization/visualizers/#1,
		class=circle visualizer,
		arg1=#1
	  },
	  new visualizer={#1}{%
		color=visualizer color, % a color setup by the style sheet
		every path/.style={fill,draw}, % fill and draw the circle by default,
	  }{, % let's ignore legends in this example
	  /data point/set=#1
  },
visualize as circle/.default=circle
}}

\datavisualization [scientific axes]
[
visualize as circle/.list={order,disorder},
legend={north east outside},
order={label in legend={text=$4^4$}},
disorder={label in legend={text=disorder}}
]
data[set=order] {
  x, y 
  -4,61
  -5,61
  -4,62
  -5,62
  -1,63
  -2,63
  -3,63
  -4,63
  -5,63
}
data[set=disorder] {
  x, y 
  -1,60
  -2,60
  -3,60
  -4,60
  -5,60
  -1,61
  -2,61
  -3,61
  -1,62
  -2,62
  -3,62
};
\end{tikzpicture}
\end{document}
